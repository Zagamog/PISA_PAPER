\documentclass[10pt]{article}
\usepackage{array}
\usepackage{longtable}
\usepackage{fullpage}
\usepackage{dcolumn}
\usepackage{times}
\usepackage[flushleft]{threeparttable}
\usepackage{tabularx}
\usepackage{booktabs}
\usepackage[12hr]{datetime}
\usepackage{longtable}
\usepackage[DIV=16]{typearea}
\usepackage{scrextend,booktabs}
\usepackage{tabulary}
\usepackage{array}
\usepackage{multirow}
\usepackage[font=small]{caption}

\begin{document}
	
\begin{table}[H]
	\footnotesize
	\def\arraystretch{0.9}
	\centering
	\caption{\textbf{Summary statistics - student attitude}}
\begin{tabulary}{1.0\textwidth}{L L C C C C}
	\hline\hline \\
	\multicolumn{2}{c}{}
	& \multicolumn{2}{c}{Dev7 countries}
	& \multicolumn{2}{c}{Vietnam}	\\
	\hline & & & & & & 
	Variable & Description & MS & Valid N &  MS & Valid N \\
	\hline \\

		\multicolumn{6}{l}{\textbf{Student Attitude towards Mathematics}}	\\ [0.3em]
		\hline \\			 
INSTMOT \textit{(r)} & Instrumental & 0.4253 & 26566 & 0.3683 & 3220 \\ 
& motivation for math & (0.8558) &  & (0.7289) &  \\ [0.3em]
INTMAT \textit{(r)} & Interest in & 0.7212 & 26634 & 0.6927 & 3219 \\ 
& mathematics & (0.8533) &  & (0.6636) &  \\ [0.3em]
SUBNORM \textit{(r)} & Subjective norms & 0.716 & 26509 & -0.0923 & 3220 \\ 
& in mathematics & (1.165) &  & (0.8395) &  \\ [0.3em]
MATHEFF \textit{(r)} & Self-Efficacy & -0.2269 & 26457 & -0.2655 & 3217 \\ 
& in mathemtatics & (0.8516) &  & (0.6363) &  \\ [0.3em]

FAILMAT \textit{(r)} & Attributions to & 0.083 & 26155 & 0.0895 & 3214 \\ 
& failure in math & (1.0312) &  & (0.6319) &  \\ [0.3em]
MATINTFC \textit{(r)} & Mathematics & 0.092 & 24827 & 0.3285 & 3181 \\ 
& intentions & (0.9837) &  & (1.0964) &  \\ [0.3em]
MATBEH \textit{(r)} & Mathematics & 0.8764 & 25899 & 0.6757 & 3211 \\ 
& behaviour & (0.9697) &  & (0.6408) &  \\ [0.3em]
PERSEV \textit{(r)} & Perseverance & 0.3387 & 25710 & 0.4475 & 3211 \\ 
& in problem solving & (0.9605) &  & (0.8767) &  \\ [0.3em]
OPENPS \textit{(r)} & Openness to & 0.1949 & 25612 & -0.6125 & 3207 \\ 
& problem solving & (0.9787) &  & (0.8708) &  \\ [0.3em]
SCMAT \textit{(r)} & Self-concept of & 0.1673 & 26222 & -0.1896 & 3249 \\ 
&  own math skills & (0.8101) &  & (0.5903) &  \\ [0.3em]
ANXMAT \textit{(r)} & Mathematics & 0.3995 & 26275 & 0.2115 & 3248 \\ 
& Anxiety & (0.7724) &  & (0.6354) &  \\ [0.3em]
BELONG \textit{(r)} & Sense of & 0.0511 & 25785 & -0.2574 & 3253 \\ 
& belonging to school & (0.9428) &  & (0.7032) &  \\ [0.3em]
		
		\hline \\
		\multicolumn{6}{l}{\textbf{Student Attitude towards School}}	\\ [0.3em]
		\hline \\		
ATSCHL \textit{(r)} & Attitude - school & 0.1616 & 25563 & 0.143 & 3246 \\ 
& learning is useful  & (0.9986) &  & (0.8648) &  \\ [0.3em]
ATTLNACT \textit{(r)} & Attitude - Trying hard & 0.1233 & 25368 & -0.535 & 3248 \\ 
& at school pays off & (0.964) &  & (0.8212) &  \\ [0.3em]
ATT\_CONTROL \textit{(r)} & Perceived control & 0.8507 & 25106 & 0.6608 & 3228 \\ 
& over grades & (0.3564) &  & (0.4735) &  \\ [0.3em]
			
\hline \\
\multicolumn{6}{l}{Notes: The variables relate to the questionnaires administered to students in the rotated booklet.}\\   
\multicolumn{6}{l}{For a more detailed description of variables, please see Table xx. Items marked with \textit{(r)} are}\\    
\multicolumn{6}{l}{taken from the rotated student questionnaire. The variable means of Dev7 and Vietnam are}\\
\multicolumn{6}{l}{statistically different at the 5\% significance level, except FAILMAT and ATTSCHL.}\\


\end{tabulary}
\end{table}
	
	
\end{document}

